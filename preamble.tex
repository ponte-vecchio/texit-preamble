%%%%%%%%%%%%%%%%%%%%%%%%%%%%%%%%%%%%%%%%%%%%%%%%%%%%%%%%%%%%%%%%%%%%%%%%%%%%%%
%      ORIGINAL AUTHOR:  LEOTHELION
%   VERSION (YYYY.MMA):  2025.12
%%%%%%%%%%%%%%%%%%%%%%%%%%%%%%%%%%%%%%%%%%%%%%%%%%%%%%%%%%%%%%%%%%%%%%%%%%%%%%

% SHORTCUTS
\edef\tmp{\catcode`@\the\catcode`@ \space}\makeatletter
\let\exaf\expandafter
\let\ifpkg\@ifpackageloaded

% AAA
\RequirePackage{amssymb, mathtools}
\RequirePackage{mleftright}
\RequirePackage{xcolor}
\RequirePackage{booktabs}
\RequirePackage{array}
\RequirePackage{tabularx}
\RequirePackage{graphicx}
\RequirePackage{hwemoji}
\DeclareUnicodeCharacter{FE0E}{}
\DeclareUnicodeCharacter{FE0F}{}

\setlength{\tabcolsep}{.6667ex}
\setlength{\fboxrule}{0.075pt}
\setlength{\parindent}{0pt}
\setlength{\hangindent}{0pt}
\setlength{\parskip}{0pt}

% FONTS
\RequirePackage[T2A, T1]{fontenc}
\RequirePackage{libertinus}
\RequirePackage[narrow,mono,var0,varqu,scaled=1.08]{zi4}

\NewDocumentCommand		\tf{}{\normalfont}
\RenewDocumentCommand	\rm{}{\rmfamily}
\RenewDocumentCommand	\ss{}{\sffamily}
\RenewDocumentCommand	\tt{}{\ttfamily}
\RenewDocumentCommand	\bf{}{\bfseries}
\RenewDocumentCommand	\it{}{\itshape}
\RenewDocumentCommand	\sl{}{\slshape}
\NewDocumentCommand		\bi{}{\bfseries\itshape}
\NewDocumentCommand		\bs{}{\bfseries\slshape}
\RenewDocumentCommand	\sc{}{\scshape}

% CJK
\RequirePackage{kotex}

% MATH
\usepackage{amsthm, amssymb}
\RequirePackage[inline]{enumitem}
\usepackage{mathtools, esvect}  % extension to amsmath
\usepackage{diffcoeff}
\usepackage{siunitx}
\usepackage{nicematrix}
\usepackage{esvect}
\usepackage{braket}
\usepackage[frenchmath,upint]{libertinust1math}
\usepackage{bm}
\usepackage[bb=boondox,cal=cm,frak=esstix,scr=boondoxupr]{mathalpha}

% Redefine \left and \right
\ExplSyntaxOn
  \cs_new_eq:NN \AMSleft\left
  \cs_new_eq:NN \AMSright\right
  \cs_undefine:N \vec
  \cs_undefine:N \left
  \cs_undefine:N \right
  \cs_new_eq:NN \left\mleft
  \cs_new_eq:NN \right\mright
\ExplSyntaxOff
% Shortcuts
\NewDocumentCommand     \vec{ +m }{\ensuremath{\bm{#1}}}
\NewDocumentCommand     \fk	{ +m }{\ensuremath{\mathfrak{#1}}}
\NewDocumentCommand     \bb	{ +m }{\ensuremath{\mathbb{#1}}}
\RenewDocumentCommand   \cal{ +m }{\ensuremath{\mathcal{#1}}}
\NewDocumentCommand     \scr{ +m }{\ensuremath{\mathscr{#1}}}

% TYP
% Typographic enhancements
\makeatletter
\RequirePackage[tracking=smallcaps, protrusion=smallcaps, letterspace=50]{microtype}
\DeclareKeys[typ]{
  breakpenalty  .tl_gset_e:N  = \typ@breakpenalty,
  breakpenalty  .initial:n    = \exhyphenpenalty,
  dashraise     .dim_gset:N   = \typ@dashraise,
  dashraise     .initial:n    = .6pt,
  dashspace     .dim_gset:N   = \typ@dashspace,
  dashspace     .initial:n    = .1em,
}
\ProcessKeyOptions[typ]
\protected\long\def\typset#1{\SetKeys[typ]{#1}}

% Dashes
\NewDocumentCommand{\typ@breakpoint}{}{\penalty\typ@breakpenalty}
\NewDocumentCommand{\typ@allowhyp}{}{\ifvmode\relax\else\nobreak\hskip\z@skip\fi}
\unless\ifdefined\allowhyphenation\let\allowhyphenation\typ@allowhyp\fi
% Hyphens
\NewDocumentCommand{\Hyphen}{ s } {%
  \begingroup%
    \raisebox{\typ@dashraise}{\char`-}%
    \IfBooleanTF{#1}%
      {\ignorespaces}{\typ@breakpoint\typ@allowhyp}%
  \endgroup%
}
\let\capitalhyphen\Hyphen

%% En Dashes
\NewDocumentCommand{\Ndash}{ s }{%
  \begingroup%
    \raisebox{\typ@dashraise}{\textendash}%
    \IfBooleanTF{#1}%
      {\ignorespaces}{\typ@breakpoint\typ@allowhyp}%
  \endgroup%
}
\let\capitalendash\Ndash
\let\Fdash\Ndash

\newcommand*{\wrap@longdashes}[3]
  {#1\hspace{\typ@dashspace}%
   #3%
   #2\hspace{\typ@dashspace}}

\NewDocumentCommand{\endash}{ s O{\z@} }
{%
  \IfBooleanTF{#1}
    {\IfNoValueTF{#2}
      {\wrap@longdashes{\nobreak}{\nobreak}{\textendash}}
      {\wrap@longdashes{\nobreak}{\nobreak}{\raisebox{#2}{\textendash}}}}
    {\IfNoValueTF{#2}
      {\wrap@longdashes{\relax}{\relax}{\textendash}}
      {\wrap@longdashes{\relax}{\relax}{\raisebox{#2}{\textendash}}}}%
   \ignorespaces%
}

\NewDocumentCommand{\Mdash}{ s }{%
  \begingroup%
    \raisebox{\typ@dashraise}{\textemdash}%
    \IfBooleanTF{#1}%
      {\ignorespaces}{\typ@breakpoint\typ@allowhyp}%
  \endgroup%
}

\NewDocumentCommand{\emdash}{ s O{\z@} }
{%
  \IfBooleanTF{#1}
    {\IfNoValueTF{#2}
      {\wrap@longdashes{\nobreak}{\nobreak}{\textemdash}}
      {\wrap@longdashes{\nobreak}{\nobreak}{\raisebox{#2}{\textemdash}}}}
    {\IfNoValueTF{#2}
      {\wrap@longdashes{\relax}{\relax}{\textemdash}}
      {\wrap@longdashes{\relax}{\relax}{\raisebox{#2}{\textemdash}}}}%
   \ignorespaces%
}

\let\capitalemdash\Mdash
\NewDocumentCommand{\Times}{}{%
  \begingroup\ifmmode
      \mathbin{\raisebox{\typ@dashraise}{$\m@th\times$}}%
    \else
      \raisebox{\typ@dashraise}{\texttimes}%
    \fi
  \endgroup%
}

% Spacing
% hairspace can be as big as 1/10em and as little as 1/24em
% we could use plus and minus trick for it to stretch and shrink as required
\NewDocumentCommand{\hairspace}{}{}


\newcommand*{\widespacestrength}{1.}
\newcommand*{\widespacescale}{1.125}
\NewDocumentCommand{\widespace}{s}
  {\IfBooleanTF{#1}%
    {\dimen0=\widespacescale\fontdimen2\font}%
    {\ifdim\fontdimen7\font=\z@
       \dimen0=\widespacescale\fontdimen2\font
     \else
       \dimen0=\dimexpr\fontdimen2\font +
               \widespacestrength\fontdimen7\font
     \fi}%
   \hskip \glueexpr\dimen0
          \@plus \widespacescale\fontdimen3\font
          \@minus \widespacescale\fontdimen4\font
   \ignorespaces}

\newcommand*{\narrowspacestrength}{.5}
\newcommand*{\narrowspacescale}{.9375}
\NewDocumentCommand{\narrowspace}{s}
  {\IfBooleanTF{#1}%
     {\dimen0=\narrowspacescale\fontdimen2\font}%
     {\ifdim\fontdimen7\font=\z@
        \dimen0=\narrowspacescale\fontdimen2\font
      \else
        \dimen0=\dimexpr\fontdimen2\font -
                \narrowspacestrength\fontdimen7\font
      \fi}%
   \hskip \glueexpr\dimen0
          \@plus \narrowspacescale\fontdimen3\font
          \@minus \narrowspacescale\fontdimen4\font
   \ignorespaces}

% Show existing definitions heretofore
\NewDocumentCommand{\ShowTypDefs}{}{%
  \def\typ@cs##1{{\ttfamily\color{magenta}\textbackslash##1}}
  % show optional args which are s (star) and o (optional)
  \def\typ@opts##1{{\ttfamily[\narrowspace{\color{blue}##1}\narrowspace]}}
  \par\noindent{\bfseries Definitions within {\ttfamily Typ}}\par\smallskip
  \typ@cs{Hyphen}\typ@opts{s}\quad
    Hyphenation that accounts for capital letters.\par\smallskip
  \typ@cs{Ndash}\typ@opts{s}\quad
    En dash that accounts for capital letters.\par\smallskip
  \typ@cs{endash}\typ@opts{s o[{\string\z@}]}\quad
    En dash with optional raise with extra kerning. The starred version makes it unbreakable.\par\smallskip
  \typ@cs{emdash}\typ@opts{s o[{\string\z@}]}\quad
    Em dash with optional raise with extra kerning. The starred version makes it unbreakable.\par\smallskip
  \typ@cs{Times}\quad
    Multiplication sign raised to the appropriate height in text and math modes.\par\smallskip
  \typ@cs{widespace}\typ@opts{s}\quad
    Wider interword space. The starred version ignores stretchability from font.\par\smallskip
  \typ@cs{narrowspace}\typ@opts{s}\quad
    Narrower interword space. The starred version ignores stretchability from font.\par
}

% LLIPSUM
%% Story.tex but Brian Eno
\NewDocumentCommand{\eno}{}{%
  \begin{minipage}{\hsize}
    \begin{center}
    {\bfseries Ambient 1: Music for Airports}

    {\scshape Genre}: Electronic

    {\scshape Label}: EG/Polydor/PVC

    {\scshape Year}: 1978
    \end{center}
  ``Sitting among the gleaming steel fixtures and softly-glowing concrete lines of the modernist {\itshape Flughafen K\"oln/Bonn\/} on a sunny Sunday morning in late 1977, en route to his homebase, the perennially nervous flier recoiled once again at the canned pop pleasantries mindless piped into such an inspired space. The music was not only an afterthought but also insulting to the idea that you would soon climb into a sleep metal tube and be propelled by engines through the sky at 40\narrowspace*{}000 feet. ``I started thinking\emdash{}{\slshape`What should we be hearing here?'\/} I thought most of all you wanted music that didn't try to pretend that you weren't going to die on the plane,'' Eno, laughing but serious.''
  \end{minipage}%
}

% DEMOBOX
% Demobox
\ExplSyntaxOn
\NewDocumentCommand{\@convert}{ +m +m O{2} }{
	\str_case:nnF {#2}{
		{ sp } { \fp_eval:n{ round(\dim_to_decimal_in_unit:nn{#1/#3}{1#2}, 0) } }
		{ pt } { \fp_eval:n{ round(\dim_to_decimal_in_unit:nn{#1   }{1#2}, 2) } }
		{ pc } { \fp_eval:n{ round(\dim_to_decimal_in_unit:nn{#1/#3}{1#2}, 3) } }
		{ in } { \fp_eval:n{ round(\dim_to_decimal_in_unit:nn{#1/#3}{1#2}, 3) } }
		{ cm } { \fp_eval:n{ round(\dim_to_decimal_in_unit:nn{#1/#3}{1#2}, 3) } }
	}{% bp, dd, cc, mm follow default
		\fp_eval:n{ round(\dim_to_decimal_in_unit:nn{(#1 / #3)}{1#2}, 2) }
	}
}
\ExplSyntaxOff

%% Demobox
\NewDocumentCommand{\demobox}{ s O{2} +m O{pt} }{%
	\setbox0=\hbox{#3}
	\setbox1=\hbox{x}% for x height
	\setbox2=\hbox{H}% for cap height
	\font\tinymono=CascadiaCode-Light-tlf-t1 at 2pt
	\font\tinymonobf=CascadiaCode-Regular-tlf-t1 at 2pt
	\DeclareRobustCommand{\@repr}[3]{%
		\begingroup\tinymono ##1%
		% typeset ':' if #1 is given
		\ifx##1\relax\else:\thinspace\fi\endgroup%
		\tinymonobf\color{magenta!80!black}\@convert{##2}{##3}%
		\thinspace##3%
	}
	\scalebox{#2}[#2]{%
		\begin{picture}(2\wd0, 1.6\ht0)
			\setlength{\unitlength}{1em}
			\setlength{\fboxrule}{0em}
			\setlength{\fboxsep}{0em}
			\linethickness{0.015em}
			\def\Ldist{0.25}
			\def\outerdist{0.45}
			% rulers
			\put(0, \ht0){\color{black!60}\line(1,0){\wd0}} % ascender
			\IfBooleanF{#1}{%
				\ifdim\dimexpr\ht0-\ht2\relax>2pt% cap-height
				\put(0, \ht2){\color{blue!40}\line(1,0){\wd0}}
				\fi
				\ifdim\dimexpr\ht0-\ht1\relax>2pt% x-height
				\put(0, \ht1){\color{green!40}\line(1,0){\wd0}}
				\fi
			}
			\ifdim\dp0>0.02em
			\put(0,-\dp0){\color{black!60}\line(1,0){\wd0}} % descender
			\fi
			\put(0, 0){\color{red!90!black!50}\line(1,0){\wd0}} % baseline
			% content
			\put(0,0){\fbox{#3}}
			% labels
			\IfBooleanF{#1}{%
				% cap height pt
				\ifdim\dimexpr\ht0-\ht2\relax>\dimexpr\fontdimen6\tinymono/6*5\relax
				\put(\wd0+\outerdist, \ht2-0.075){\@repr{caph}{\ht2}{#4}}
				{\linethickness{0.01em}
					\put(\wd0+0.19, \ht1){\color{red!50!blue!90!black!50}\line(0,1){\ht2-\ht1}}
					\put(\wd0+0.15, \ht1){\color{red!90!black!66}\line(1,0){0.08}}
					\put(\wd0+0.15, \ht2){\color{blue!40!black}\line(1,0){0.08}}
				}
				\fi
				% x-height pt
				\ifdim\dimexpr\ht0-\ht1\relax>\dimexpr\fontdimen6\tinymono/6*5\relax
				\put(\wd0+\outerdist, 0.5\ht1-0.05){\@repr{xhgt}{\ht1}{#4}}
				{\linethickness{0.01em}
					\put(\wd0+0.19, 0){\color{red!50!green!90!black!50}\line(0,1){\ht1}}
					\put(\wd0+0.15, 0){\color{red!90!black!66}\line(1,0){0.08}}
					\put(\wd0+0.15, \ht1){\color{green!60!black}\line(1,0){0.08}}
				}
				\fi
			}
			% depth pt
			\ifdim\dp0>\dimexpr\fontdimen6\font/13\relax%
			\put(\wd0+\outerdist, -\dp0-0.05){\@repr{dpth}{\dp0}{#4}}
			{\linethickness{0.01em}\color{black!33}%
				\put(\wd0+0.09, -\dp0){\line(0,1){\dp0}}
				\put(\wd0+0.05, 0){\line(1,0){0.08}}
				\put(\wd0+0.05, -\dp0){\line(1,0){0.08}}
			}
			\fi
			% height pt
			\put(\wd0+\outerdist, \ht0){\@repr{hght}{\ht0}{#4}}{%
				\linethickness{0.01em}\color{black!33}%
				\put(\wd0+0.05, 0){\line(1,0){0.08}}
				\put(\wd0+0.05, \ht0){\line(1,0){0.08}}
				\put(\wd0+0.09, 0){\line(0,1){\ht0}}
			}
			% width pt
			% label
			\put(0, 1\ht0){\makebox(\wd0, 0.75\ht1+0.25)[c]{\@repr{}{\wd0}{#4}}}{%
				\linethickness{0.01em}\color{black!33}%
				\put(0, \ht0+0.075){\line(1,0){\wd0}}
				\put(0, \ht0+0.035){\line(0,1){0.08}}
				\put(\wd0, \ht0+0.035){\line(0,1){0.08}}
			}
		\end{picture}
	}
}

% CHEMISTRY
\usepackage[version=4]{mhchem}
\usepackage{chemfig}
\renewcommand*\printatom[1]{\footnotesize\ensuremath{\mathrm{#1}}}
\setchemfig{%
	atom sep = 1.75em,
	stack sep = 0.3em,% dim for chemabove and below
	bond offset = 1.25pt,
	double bond sep = .25em
}
\definecolor{cmap1}{HTML}{DD6E79} % #DD6E79
\definecolor{cmap2}{HTML}{5076A6} % #5076A6
\definecolor{cmap3}{HTML}{9D3C75} % #9D3C75
\definecolor{cmap4}{HTML}{AAAA35} % #AAAA35
\definecolor{cmap5}{HTML}{43863E} % #43863E
\colorlet{cmap6}{cmap3!55!cmap5}  % #9D3C75!55!43863E
\tikzset{%
	atom/.style={draw, circle, minimum size=1.25em, inner sep=0pt},
	vibr/.style={-Latex, thick}
}

% TEXIT
\makeatletter

\let\sa@cls@afterbegindocument\relax
\let\sa@cls@beforeenddocument\relax

\def\texit@version{1.2-rc20-embedded}



\ifdefined\directlua
\RequirePackage{luatex85}
\fi



\DeclareKeys[texit]{
  pagecolor.tl_gset_e:N               = \texit@pagecolor,
  pagecolor.initial:n                 = 1a1a1e,
  pagegradient.tl_gset_e:N            = \texit@pagegradient,
  pagegradientangle.tl_gset_e:N       = \texit@pagegradientangle,
  pagegradientangle.initial:n         = 0,
  textcolor.tl_gset_e:N               = \texit@textcolor,
  textcolor.initial:n                 = ffffff,
  textgradient.tl_gset_e:N            = \texit@textgradient,
  textgradientangle.tl_gset_e:N       = \texit@textgradientangle,
  textgradientangle.initial:n         = 0,
  textgradientglow.legacy_if_gset:n   = texit@textgradientglow,
  singlepage.legacy_if_gset_inverse:n = texit@singlepage,
  singlepage.usage                    = preamble,
  multipage.legacy_if_gset:n          = texit@singlepage,
  multipage.usage                     = preamble,
  minpagewidth.dim_gset:N             = \texit@minpagewidth,
  minpagewidth.initial:n              = \z@,
  minhsize.dim_gset:N                 = \texit@minhsize,
  minhsize.initial:n                  = \z@,
  maxhsize.dim_gset:N                 = \texit@maxhsize,
  maxhsize.initial:n                  = 2000\p@,
  border.tl_gset_e:N                  = \texit@border@,
  extraborders.tl_gset_e:N            = \texit@extraborders@,
  debug.legacy_if_gset:n              = texit@debug,
}
\ProcessKeyOptions[texit]
\protected\long\def\texitset#1{%
  \SetKeys[texit]{#1}%
  \texit@processgradients
  \texit@processcolors
}
\let\leosays=\texitset

% \LoadClass{article}
\let\texit@border@=\empty
\texit@singlepagetrue


\parindent=0pt

\textheight=\maxdimen
\textwidth=345pt

\pdfpageheight=10pt
\pdfpagewidth=110pt

\font\texit@debugfont=phvr7t at 2.5pt
\font\texit@debugfontbold=phvb7t at 2.5pt



% Parameters

\newif\iftexit@pagetrans
\newif\iftexit@pagegradient
\newif\iftexit@textgradient
\newdimen\texit@border

\protected\def\texit@processcolors{%
  \def\@tempa{trans}%
  \global \ifx\texit@pagecolor\@tempa \texit@pagetranstrue \else \texit@pagetransfalse \fi
  \texit@remhash\texit@pagecolor
  \texit@remhash\texit@textcolor
  \@ifpackageloaded{xcolor}{%
    \texit@definexcolor{texittextcolor}\texit@textcolor
    \global\expandafter\let\expandafter\default@color
    \csname \string\color @\iftexit@textgradient white\else texittextcolor\fi\endcsname
    \global\let\current@color\default@color
    \global\expandafter\let \csname \string\color @defaultcolor\endcsname \default@color
    \colorlet{.}{defaultcolor}%
    \global\let\XC@current@color=\default@color
    \global\let\XC@default@color=\default@color
    \iftexit@textgradient
    \global\chardef\main@pdfcolorstack=\pdfcolorstackinit{1 g 1 G}%
    \else
    \texit@parsecolor\texit@textcolor
    \global\chardef\main@pdfcolorstack=\pdfcolorstackinit{\texit@temp}%
    \fi
    \iftexit@pagetrans \else
    \texit@definexcolor{texitpagecolor}\texit@pagecolor
    \@ifpackageloaded{tikz-cd}{%
      \iftexit@textgradient
      \tikzcdset{background color=black}%
      \else
      \tikzcdset{background color=texitpagecolor}%
      \fi
    }{}%
    \fi
  }{}%
}

\protected\def\texit@processgradients{%
  \global \ifx\texit@pagegradient\empty
  \texit@pagegradientfalse
  \else
  \texit@pagegradienttrue
  \texit@remhash\texit@pagegradient
  \fi
  \global \ifx\texit@textgradient\empty
  \texit@textgradientfalse
  \else
  \texit@textgradienttrue
  \texit@remhash\texit@textgradient
  \fi
}

\protected\def\texit@processall{%
  \texit@processgradients
  \texit@processcolors
}

% Border width function used if border option not given
% (default: max(5, min(15, d/24 - 5))pt where d = max(ht255+dp255, wd255))
\protected\gdef\texit@border@calc#1{%
  \texit@border=\dimexpr\ht#1+\dp#1\relax
  \ifdim\wd#1>\texit@border \texit@border=\wd#1\fi
  \texit@border=\dimexpr\texit@border/24-5\p@ \relax
  \ifdim\texit@border<5\p@ \texit@border=5\p@ \fi
  \ifdim\texit@border>15\p@ \texit@border=15\p@ \fi
}%

% Parameter that controls greediness of horizontal cropping (lower = more greedy, must be between 1 and 10000)
\mathchardef\texit@realwd@minhbadness=1000

\AtBeginDocument{\texit@processall}



% Output Routine

\ifcsname tex_shipout:D\endcsname
\expandafter\let\expandafter\texit@shipout@final\csname tex_shipout:D\endcsname
\else
\let\texit@shipout@final=\shipout
\fi

% Holds final output page
\newbox\texit@pagebox

\newtoks\texit@underlay
\newtoks\texit@overlay

\newdimen\texit@extraborder@T
\newdimen\texit@extraborder@L
\newdimen\texit@extraborder@B
\newdimen\texit@extraborder@R

\newdimen\texit@pagebox@realwd
\newdimen\texit@boxcclv@realwd
\newdimen\texit@topins@realwd
\newdimen\texit@botins@realwd
\newdimen\texit@footins@realwd

% Marker for OTR to discard the current page
\newcount\texit@discardpage \texit@discardpage=-"7AFF

% Marker for top of vlist
\mathchardef\texit@realwd@marker="7AFE

% Generic marker to force retention of full width
\mathchardef\texit@realwd@forcehsize="7AFD

% Marker for display equations
\mathchardef\texit@realwd@postdisplaypenalty="7AFC

% If previous box was a display equation
\newif\iftexit@realwd@isdisplay

\AddToHook{begindocument/end}{
  \null \vfil \penalty\texit@discardpage
  \topskip=0pt
  \postdisplaypenalty=\texit@realwd@postdisplaypenalty
  \penalty\texit@realwd@marker
}
\AddToHook{enddocument}{
  \iftexit@singlepage
  \ifdim\pagetotal<1sp \nointerlineskip\null \fi
  \def\clearpage{\penalty-\@MM}%
  \fi
}

\output={%
  \iftexit@singlepage
  \ifnum\outputpenalty=-\@MM
  \let\par=\@@par
  \texit@shipout
  \global\output={\deadoutput}%
  \else
  \ifnum\outputpenalty=\texit@discardpage
  \setbox\texit@pagebox=\box\@cclv
  \else
  \unvbox\@cclv
  \fi
  \fi
  \else
  \ifnum\outputpenalty=\texit@discardpage
  \setbox\texit@pagebox=\box\@cclv
  \else
  \ifdim\wd\@cclv=\z@
  \setbox\texit@pagebox=\box\@cclv
  \else
  \let\par=\@@par
  \texit@shipout
  \fi
  \fi
  \fi
}
\protected\def\deadoutput{\setbox\texit@pagebox=\box\@cclv}

\protected\def\texit@shipout{%
  \ifvoid\texit@topins \else \setbox\texit@topins=\vbox{\unvbox\texit@topins \unskip}\fi
  \ifvoid\texit@botins \else \setbox\texit@botins=\vbox{\unvbox\texit@botins \unskip}\fi
  \texit@shipout@setrealwd
  \setbox\texit@pagebox=\vbox{%
    \iftexit@textgradient \else
    \texit@parsecolor\texit@textcolor
    \pdfliteral{\texit@temp}%
    \fi
    \ifvoid\texit@topins \else
    \unvbox\texit@topins
    \vskip \skip\texit@topins
    \fi
    \unvbox\@cclv
    \ifvoid\texit@botins \else
    \vskip \skip\texit@botins
    \unvbox\texit@botins
    \fi
    \ifvoid\footins \else
    \vskip \skip\footins
    \footnoterule
    \unvbox\footins
    \fi
  }%
  \wd\texit@pagebox=\texit@pagebox@realwd
  \ifx\texit@border@\empty
  \texit@border@calc\texit@pagebox
  \else
  \texit@border \dimexpr\texit@border@\relax
  \fi
  \ifx\texit@extraborders@\empty \else
  \expandafter\texit@extraborders@parse\texit@extraborders@\@nil
  \fi
  %
  \voffset=-1in
  \hoffset=\voffset
  \pdfpageheight=\dimexpr\ht\texit@pagebox+\dp\texit@pagebox+2\texit@border+\texit@extraborder@T+\texit@extraborder@B +2\p@ \relax
  \pdfpagewidth=\dimexpr\texit@pagebox@realwd+2\texit@border+\texit@extraborder@L+\texit@extraborder@R +2\p@ \relax
  \ifdim\pdfpagewidth<\texit@minpagewidth
  \pdfpagewidth=\texit@minpagewidth
  \fi
  \set@display@protect
  \texit@shipout@final \vbox{\kern\p@ \moveright\p@ \vbox to\pdfpageheight{%
      \nointerlineskip \vbox to\z@{%
        \set@typeset@protect
        \hsize=\dimexpr\texit@pagebox@realwd+2\texit@border+\texit@extraborder@L+\texit@extraborder@R \relax
        \vsize=\pdfpageheight
        \the\texit@underlay
        \vss
      }%
      \nointerlineskip \moveright\dimexpr\texit@border+\texit@extraborder@L \vbox to\z@{%
        \kern\texit@border \kern\texit@extraborder@T
        \iftexit@textgradient
        \texit@gradient@parse\texit@textgradient
        \def\texit@debug@node@color{1 1 0}%
        \texit@gradientbox\texit@pagebox\texit@textgradientangle
        % On top of existing debug overlay
        \global\setbox\texit@debug@pagebox=\vbox{%
          \vbox to\z@{\box\texit@debug@pagebox \vss}%
          \unvbox\texit@debug@gradientbox
        }%
        \else
        \box\texit@pagebox
        \fi
        \vss
      }%
      \nointerlineskip \vbox to\z@{%
        \set@typeset@protect
        \hsize=\dimexpr\texit@pagebox@realwd+2\texit@border+\texit@extraborder@L+\texit@extraborder@R \relax
        \vsize=\pdfpageheight
        \the\texit@overlay
        \vss
      }%
      \iftexit@debug
      \nointerlineskip \moveright\dimexpr\texit@border+\texit@extraborder@L
      \vbox to\z@{\kern\texit@border \kern\texit@extraborder@T \box\texit@debug@pagebox \vss}%
      \fi
      \vfil
      \iftexit@debug \texit@debug@footline \fi
  }}%
}

\texit@underlay={%
  \texit@paintpage
}

\def\texit@extraborders@parse#1,#2,#3,#4\@nil{%
  \global\texit@extraborder@T=\dimexpr#1+\z@\relax
  \global\texit@extraborder@L=\dimexpr#2+\z@\relax
  \global\texit@extraborder@B=\dimexpr#3+\z@\relax
  \global\texit@extraborder@R=\dimexpr#4+\z@\relax
}

\protected\def\texit@shipout@setrealwd{%
  \ifvoid\texit@topins \else
  \texit@realwd\texit@topins\texit@topins@realwd
  \edef\@tempb{\the\texit@debug@boxes}%
  \ifdim\texit@topins@realwd>\texit@pagebox@realwd
  \texit@pagebox@realwd=\texit@topins@realwd
  \fi
  \fi
  \texit@realwd\@cclv\texit@boxcclv@realwd
  \edef\@tempa{\the\texit@debug@boxes}%
  \texit@pagebox@realwd=\texit@boxcclv@realwd
  \ifvoid\texit@botins \else
  \texit@realwd\texit@botins\texit@botins@realwd
  \edef\@tempc{\the\texit@debug@boxes}%
  \ifdim\texit@botins@realwd>\texit@pagebox@realwd
  \texit@pagebox@realwd=\texit@botins@realwd
  \fi
  \fi
  \ifvoid\footins \else
  \texit@realwd\footins\texit@footins@realwd
  \edef\@tempd{\the\texit@debug@boxes}%
  \ifdim\texit@footins@realwd>\texit@pagebox@realwd
  \texit@pagebox@realwd=\texit@footins@realwd
  \fi
  \ifdim\texit@pagebox@realwd<\footnoterulewd
  \texit@pagebox@realwd=\footnoterulewd
  \fi
  \fi
  \iftexit@debug
  \setbox\texit@debug@pagebox=\vbox{%
    \ifvoid\texit@topins \else
    \setbox\z@=\vbox{\unvcopy\texit@topins}%
    \wd\z@=\texit@topins@realwd
    \texit@debug@draw@page\z@\@tempb{topins}%
    \kern\ht\z@ \kern\dp\z@
    \vskip \skip\texit@topins
    \fi
    \setbox\z@=\vbox{\unvcopy\@cclv}%
    \wd\z@=\texit@boxcclv@realwd
    \texit@debug@draw@page\z@\@tempa{box255}%
    \kern\ht\z@ \kern\dp\z@
    \ifvoid\texit@botins \else
    \vskip \skip\texit@botins
    \setbox\z@=\vbox{\unvcopy\texit@botins}%
    \wd\z@=\texit@botins@realwd
    \texit@debug@draw@page\z@\@tempb{botins}%
    \kern\ht\z@ \kern\dp\z@
    \fi
    \ifvoid\footins \else
    \vskip \skip\footins
    \vskip \footnoteruleht
    \setbox\z@=\vbox{\unvcopy\footins}%
    \wd\z@=\texit@footins@realwd
    \texit@debug@draw@page\z@\@tempd{footins}%
    \fi
  }%
  \fi
}

% Compute real width of #1 and store in #2
\protected\def\texit@realwd#1#2{%
  #2=\wd#1%
  \dimen@=\z@
  \texit@debug@boxes={}%
  \setbox\z@=\vbox{%
    \hbadness=\@M
    \penalty\texit@realwd@marker
    \unvcopy#1%
    \def\next{\texit@realwd@sift#2}\next
  }%
  \ifdim#2<\texit@minhsize \global#2=\texit@minhsize \fi
  \ifdim#2>\texit@maxhsize \global#2=\texit@maxhsize \fi
}

\def\texit@realwd@sift#1{%
  \texit@realwd@isdisplayfalse
  \texit@realwd@sift@superunskip
  \setbox\z@=\lastbox
  \ifvoid\z@
  \ifnum\lastpenalty=\texit@realwd@marker
  \texit@realwd@sift@end{#1}%
  \else
  \texit@realwd@sift@endA{#1}%
  \fi
  \let\next\relax
  \else
  \iftexit@realwd@isdisplay
  \iftexit@debug \texit@debug@draw@box{1 0 0}{1 0 0 1 \strip@pt\dimexpr(\hsize-\wd\z@)/2\relax\space 0 cm}\z@ \fi
  \ifdim\dimen@<\hsize \dimen@=\hsize \fi
  \else
  \ifhbox\z@
  \setbox\tw@=\hbox to\wd\z@{\unhcopy\z@}%
  \count@=\badness
  \ifnum\count@=1000000 % overfull box
  \setbox\tw@=\hbox{\unhbox\z@ \texit@superunskip}%
  \ifdim\wd\tw@>\dimen@
  \dimen@=\wd\tw@
  \fi
  \iftexit@debug \texit@debug@draw@box{0 1 0}{}\tw@ \fi
  \else
  \setbox\tw@=\hbox to\wd\z@{\unhcopy\z@ \texit@superunskip}%
  \ifnum\numexpr\badness-\count@>\texit@realwd@minhbadness
  \setbox\tw@\hbox{\unhbox\z@ \texit@superunskip}%
  \ifdim\wd\tw@>\dimen@
  \dimen@=\wd\tw@
  \fi
  \iftexit@debug \texit@debug@draw@box{0 1 0}{}\tw@ \fi
  \else
  \ifdim\wd\z@>\dimen@
  \dimen@=\wd\z@
  \fi
  \iftexit@debug \texit@debug@draw@box{0 1 0}{}\z@ \fi
  \fi
  \fi
  \else
  \ifdim\dimexpr\wd\z@+1sp>\dimexpr\texit@maxhsize\relax
  \let\next=\relax
  \else
  \ifdim\wd\z@>\dimen@
  \dimen@=\wd\z@
  \fi
  \fi
  \iftexit@debug \texit@debug@draw@box{0 0 1}{}\z@ \fi
  \fi
  \fi
  \ifdim\dimexpr\dimen@+1sp>\texit@maxhsize
  \texit@realwd@sift@endA{#1}%
  \let\next\relax
  \fi
  \fi
  \next
}

% Remove all trailing skips, kerns, and penalties
\protected\def\texit@superunskip{%
  \ifcase\lastnodetype
  % 0: char node
  \or % 1: hlist node
  \or % 2: vlist node
  \or % 3: rule node
  \or % 4: ins node
  \or % 5: mark node
  \or % 6: adjust node
  \or % 7: ligature node
  \or % 8: disc node
  \or % 9: whatsit node
  \or % 10: math node
  \or % 11: glue node
  \unskip
  \expandafter\texit@superunskip
  \or % 12: kern node
  \unkern
  \expandafter\texit@superunskip
  \or % 13: penalty node
  \unpenalty
  \expandafter\texit@superunskip
  \or % 14: unset node
  \or % 15: math mode nodes
  \fi
}
% Version of \texit@superunskip with special message handling and saves total distance in \dimen@i
\protected\def\texit@realwd@sift@superunskip{\dimen@i=\z@ \texit@realwd@sift@dosuperunskip}
\def\texit@realwd@sift@dosuperunskip{%
  \let\next@=\relax
  \ifcase\lastnodetype
  % 0: char node
  \or % 1: hlist node
  \or % 2: vlist node
  \or % 3: rule node
  \or % 4: ins node
  \or % 5: mark node
  \or % 6: adjust node
  \or % 7: ligature node
  \or % 8: disc node
  \or % 9: whatsit node
  \or % 10: math node
  \or % 11: glue node
  \advance\dimen@i \lastskip
  \unskip
  \let\next@=\texit@realwd@sift@dosuperunskip
  \or % 12: kern node
  \advance\dimen@i \lastkern
  \unkern
  \let\next@=\texit@realwd@sift@dosuperunskip
  \or % 13: penalty node
  \ifnum\lastpenalty=\texit@realwd@marker \else
  \ifnum\lastpenalty=\texit@realwd@forcehsize
  \ifdim\dimen@<\hsize \dimen@=\hsize \fi
  \unpenalty
  \let\next@=\texit@realwd@sift@dosuperunskip
  \else
  \ifnum\lastpenalty=\texit@realwd@postdisplaypenalty
  \unpenalty \unskip \unpenalty
  \texit@realwd@isdisplaytrue
  \else
  \unpenalty
  \let\next@=\texit@realwd@sift@dosuperunskip
  \fi
  \fi
  \fi
  \or % 14: unset node
  \or % 15: math mode nodes
  \fi
  \next@
}
\def\texit@realwd@sift@end#1{%
  \global#1\dimen@
}
\def\texit@realwd@sift@endA#1{%
  \ifdim\dimen@>#1
  \global#1\dimen@
  \fi
}



% Colors

% Convert HTML or named color to pdf color operator and save results in \texit@temp
\protected\def\texit@parsecolor#1{%
  \begingroup
  \escapechar=`\\%
  \ifcsname \string\color @#1\endcsname
  \long\def\xcolor@##1##2##3##4{##2}%
  \xdef\texit@temp{\csname \string\color @#1\endcsname}%
  \else
  \uppercase\expanded{{\edef\noexpand\texit@temp{\noexpand\texit@parsecolorA#1}}}%
  \xdef\texit@temp{\texit@temp rg \texit@temp RG}%
  \fi
  \endgroup
}
\def\texit@parsecolorA#1#2#3#4#5#6{%
  \strip@pt\dimexpr"#1#2\p@/"FF\relax\space
  \strip@pt\dimexpr"#3#4\p@/"FF\relax\space
  \strip@pt\dimexpr"#5#6\p@/"FF\relax\space
}
% Convert HTML or named color to tuple of rgb values and save results in \texit@temp
\protected\def\texit@parsecolor@rgb#1{%
  \begingroup
  \escapechar=`\\%
  \ifcsname \string\color @#1\endcsname
  \convertcolorspec{named}{#1}{HTML}\texit@temp
  \xdef\texit@temp{\expandafter\texit@parsecolorA \texit@temp}%
  \else
  \uppercase\expanded{{\xdef\noexpand\texit@temp{\noexpand\texit@parsecolorA#1}}}%
  \fi
  \endgroup
}

% (xcolor) Define named color #1 using HTML or named color #2
\protected\def\texit@definexcolor#1#2{%
  \begingroup \escapechar=`\\%
  \expandafter \endgroup
  \ifcsname \string\color @#2\endcsname
  \colorlet{#1}{#2}%
  \else
  \definecolor{#1}{HTML}{#2}%
  \fi
}

% Remove hashes (and spaces) in-place from macro
\protected\def\texit@remhash#1{%
  \xdef#1{\expandafter\texit@remhashA#1{\expandafter\@gobble}}%
}
\long\def\texit@remhashA#1{%
  \if###1\else #1\fi
  \texit@remhashA
}

\protected\def\texit@paintpage{%
  \iftexit@pagetrans \else
  \iftexit@pagegradient
  \nointerlineskip \vbox to\z@{%
    \setbox\z@=\vbox{%
      \pdfliteral{1 g 1 G}%
      \hrule height\vsize width\hsize
      \vss
    }%
    \texit@gradient@parse\texit@pagegradient
    \texit@textgradientglowfalse
    \texit@gradientbox@overshoot=\z@
    \def\texit@debug@node@color{1 .5 0}%
    \texit@gradientbox\z@\texit@pagegradientangle
    % Below existing debug overlay
    \global\setbox\texit@debug@pagebox=\vbox{%
      \vbox to\z@{\kern-\texit@border \kern-\texit@extraborder@T \moveleft\dimexpr\texit@border+\texit@extraborder@L \box\texit@debug@gradientbox \vss}%
      \unvbox\texit@debug@pagebox
    }%
    \vss
  }%
  \else
  \pdfsave
  \texit@parsecolor\texit@pagecolor
  \pdfliteral{\texit@temp}%
  \nointerlineskip \vbox to\z@{%
    \hrule height\vsize width\hsize
    \vss
  }%
  \pdfrestore
  \fi
  \fi
}

\protected\def\texit@gradientbox#1#2{%
  \if\relax \ifdim\wd#1=\z@.\fi \ifdim\dimexpr\ht#1+\dp#1=\z@.\fi \relax
  \vbox\bgroup
  \expanded{\noexpand\texit@gradientbox@setends{#1}{\the\numexpr#2-(#2/360)*360}}%
  \iftexit@debug
  \global\setbox\texit@debug@gradientbox=\vbox{\texit@gradientbox@startx}{\texit@gradientbox@starty}%
    \texit@debug@draw@end\texit@debug@node@color{100\%}{\texit@gradientbox@endx}{\texit@gradientbox@endy}%
    \pdfliteral{%
      q
      .996264 0 0 .996264 0 0 cm
      \texit@debug@node@color\space RG
      .3 w
      [1.25] 0 d
      \expandafter\texit@rempt \texit@gradientbox@startx\space
      \expandafter\texit@rempt \texit@gradientbox@starty\space
      m
      \expandafter\texit@rempt \texit@gradientbox@endx\space
      \expandafter\texit@rempt \texit@gradientbox@endy\space
      l S
      Q
    }%
  }%
  \fi
  \setbox2=\hbox to\dimexpr\wd#1+2\texit@gradientbox@overshoot{%
    \kern\texit@gradientbox@overshoot
    \vbox to\dimexpr\ht#1+\dp#1+2\texit@gradientbox@overshoot{%
      \kern\texit@gradientbox@overshoot
      \pdfliteral{q .996264 0 0 .996264 0 0 cm /Sh sh Q}%
      \vfil
    }%
    \hfil
  }%
  \setbox#1=\hbox to\dimexpr\wd#1+2\texit@gradientbox@overshoot{%
    \kern\texit@gradientbox@overshoot
    \vbox to\dimexpr\ht#1+\dp#1+2\texit@gradientbox@overshoot{%
      \kern\texit@gradientbox@overshoot
      \texit@gradientboxA{#1}%
      \vfil
    }%
    \hfil
  }%
  \immediate\pdfxform#1
  \setbox4=\hbox{\pdfrefxform\pdflastxform}%
  \immediate\pdfxform attr{/Group << /S /Transparency /CS /DeviceGray >>}4
  \count@=\pdflastxform
  \def\do##1##2{%
    \texit@parsecolor@rgb{##2}%
    \ifnum##1=1
    \edef\next{ << /FunctionType 2 /Domain [0 1] /C0 [\texit@temp] /C1 }%
    \else
    \ifnum##1=\texit@gradient@count
    \edef\next{\next [\texit@temp] /N 1 >> }%
    \else
    \edef\next{\next [\texit@temp] /N 1 >> << /FunctionType 2 /Domain [0 1] /C0 [\texit@temp] /C1 }%
    \fi
    \fi
  }%
  \texit@gradient@colors
  \immediate\pdfxform resources{\texit@gradient@shader}2
  \setbox6=\hbox{\pdfliteral direct{/smask gs}\pdfrefxform\pdflastxform}%
  \pdfobj reserveobjnum\relax
  \immediate\pdfobj useobjnum\pdflastobj{\texit@gradient@egsresource}%
  \immediate\pdfxform resources{/ExtGState \the\pdflastobj\space 0 R}6
  \kern-\texit@gradientbox@overshoot
  \nointerlineskip \hbox{\kern-\texit@gradientbox@overshoot \pdfrefxform\pdflastxform \kern-\texit@gradientbox@overshoot}%
  \kern-\texit@gradientbox@overshoot
  \egroup
  \else \box#1
  \fi
}
\def\texit@gradientboxA#1{%
  \iftexit@textgradientglow
  \texit@gradientboxB#1.0149;4.5;%
  \texit@gradientboxB#1.0291;3.5;%
  \texit@gradientboxB#1.0406;3;%
  \texit@gradientboxB#1.0567;2.5;%
  \texit@gradientboxB#1.0791;2;%
  \texit@gradientboxB#1.0934;1.75;%
  \texit@gradientboxB#1.1104;1.5;%
  \texit@gradientboxB#1.1304;1.25;%
  \texit@gradientboxB#1.1540;1;%
  \texit@gradientboxB#1.1820;.75;%
  \texit@gradientboxB#1.2150;.5;%
  \texit@gradientboxB#1.2539;.25;%
  \fi
  \pdfliteral{1 g 1 G}\nointerlineskip\box#1
}
\def\texit@gradientboxB#1.#2;#3;{%
  \setbox\z@=\vbox{%
    \pdfliteral{q 1 g 1 G 1 j 1 J 1 Tr #3 w}%
    \vbox to\z@{\kern\texit@gradientbox@overshoot
      \moveright\texit@gradientbox@overshoot \copy#1 \vss}%
    \pdfliteral{Q}%
    \vbox to\dimexpr\ht#1+\dp#1+2\texit@gradientbox@overshoot{\hbox to\dimexpr\wd#1+2\texit@gradientbox@overshoot{}\vss}%
  }%
  \setbox\tw@=\vbox{%
    \pdfliteral{q .#2 g .#2 G}%
    \vbox to\z@{\hrule height\dimexpr\ht#1+\dp#1+2\texit@gradientbox@overshoot
      width\dimexpr\wd#1+2\texit@gradientbox@overshoot \vss}%
    \pdfliteral{Q}%
    \vbox to\dimexpr\ht#1+\dp#1+2\texit@gradientbox@overshoot{\hbox to\dimexpr\wd#1+2\texit@gradientbox@overshoot{}\vss}%
  }%
  \nointerlineskip \moveleft\texit@gradientbox@overshoot
  \vbox to\z@{\kern-\texit@gradientbox@overshoot \texit@mask\z@\tw@ \vss}%
}

% Use vbox#1 as a clipping mask over vbox#2. The two boxes should have matching dimensions.
\protected\def\texit@mask#1#2{%
  \begingroup
  \immediate\pdfxform#1
  \setbox\z@=\hbox{\pdfrefxform\pdflastxform}%
  \immediate\pdfxform attr{/Group << /S /Transparency /CS /DeviceGray >>}\z@
  \count@=\pdflastxform
  \immediate\pdfxform#2
  \setbox2=\hbox{\pdfliteral direct{/smask gs}\pdfrefxform\pdflastxform}%
  \pdfobj reserveobjnum\relax
  \immediate\pdfobj useobjnum\pdflastobj{
    << /smask << /SMask <<
    /S /Luminosity /G \the\count@\space 0 R
    >> >> >>
  }%
  \immediate\pdfxform resources{/ExtGState \the\pdflastobj\space 0 R}2
  \pdfrefxform\pdflastxform
  \endgroup
}

\newdimen\texit@gradientbox@overshoot
\texit@gradientbox@overshoot=10\p@

\def\texit@gradient@shader{%
  /Shading <<
  /Sh <<
  /ShadingType 2
  /ColorSpace /DeviceRGB
  /Domain [0 \the\numexpr\texit@gradient@count-1\relax]
  /Coords [
  \expandafter\texit@rempt \texit@gradientbox@startx\space
  \expandafter\texit@rempt \texit@gradientbox@starty\space
  \expandafter\texit@rempt \texit@gradientbox@endx\space
  \expandafter\texit@rempt \texit@gradientbox@endy\space
  ]
  /Function <<
  /FunctionType 3
  /Domain [0 \the\numexpr\texit@gradient@count-1\relax]
  /Functions [\next]
  /Bounds [\texit@gradient@bounds 1\texit@gradient@count]
  /Encode [\texit@gradient@encode 1\texit@gradient@count]
  >>
  /Extend [true true]
  >>
  >>
}
\def\texit@gradient@bounds#1#2{%
  \ifnum#1<\numexpr#2-1\relax
  #1 %
  \expandafter\texit@gradient@bounds\expandafter{\the\numexpr #1+1\relax}{#2}%
  \fi
}
\def\texit@gradient@encode#1#2{%
  \ifnum#1<#2
  0 1 %
  \expandafter\texit@gradient@encode\expandafter{\the\numexpr #1+1\relax}{#2}%
  \fi%
}
\def\texit@gradient@egsresource{%
  << /smask << /SMask <<
  /S /Luminosity /G \the\count@\space 0 R
  >> >> >>
}

% Parse a comma-separated list of hex values to store in \texit@gradient@colors, and set \texit@gradient@count
\protected\def\texit@gradient@parse#1{%
  \begingroup
  \count@=\z@
  \let\do\relax
  \let\texit@gradient@colors\empty
  \expanded{\noexpand\@for\noexpand\next:=#1}\do{%
    \if\relax\detokenize\expanded{{\expandafter\@gobble\next.}}\relax \else
    \advance\count@ 1
    \xdef\texit@gradient@colors{\texit@gradient@colors \do{\the\count@}{\romannumeral-`\0\next}}%
    \fi
  }%
  \ifnum\count@=1
  \count@=2
  \def\do##1##2{\noexpand\do1{##2}\noexpand\do2{##2}}%
  \xdef\texit@gradient@colors{\texit@gradient@colors}%
  \fi
  \ifnum\count@<2 \ClassError{texit}{Malformed gradient}{}\fi
  \xdef\texit@gradient@count{\the\count@}%
  \endgroup
}

% Compute endpoints of the gradient
\protected\def\texit@gradientbox@setends#1#2{%
  \dimen@i=.5\wd#1
  \dimen@ii=\dimexpr.5\ht#1+.5\dp#1\relax
  \ifnum#2<-90
  \texit@gradientbox@setendsA{#2}++%
  \else\ifnum#2<0
  \texit@gradientbox@setendsA{#2}-+%
  \else\ifnum#2<90
  \texit@gradientbox@setendsA{#2}--%
  \else
  \texit@gradientbox@setendsA{#2}+-%
  \fi \fi \fi
}
\def\texit@gradientbox@setendsA#1#2#3{%
  \texit@gradientbox@setendsB
  { cos(#1*deg)*(#2\dimen@i*cos(#1*deg)+#3\dimen@ii*sin(#1*deg))+\dimen@i}%
  { sin(#1*deg)*(#2\dimen@i*cos(#1*deg)+#3\dimen@ii*sin(#1*deg))-\dimen@ii}%
  {-cos(#1*deg)*(#2\dimen@i*cos(#1*deg)+#3\dimen@ii*sin(#1*deg))+\dimen@i}%
  {-sin(#1*deg)*(#2\dimen@i*cos(#1*deg)+#3\dimen@ii*sin(#1*deg))-\dimen@ii}%
}
\begingroup \lccode`!=`p \lccode`?=`t \ExplSyntaxOn
\lowercase{\endgroup
  \def\texit@gradientbox@setendsB#1#2#3#4{%
    \edef\texit@gradientbox@startx{\fp_to_decimal:n{round(#1,5)}!?}%
    \edef\texit@gradientbox@starty{\fp_to_decimal:n{round(#2,5)}!?}%
    \edef\texit@gradientbox@endx  {\fp_to_decimal:n{round(#3,5)}!?}%
    \edef\texit@gradientbox@endy  {\fp_to_decimal:n{round(#4,5)}!?}%
  }
  \long\def\texit@rempt#1!?{#1}%
}



% Debug visuals

\newbox\texit@debug@pagebox
\newbox\texit@debug@gradientbox
\newtoks\texit@debug@boxes

\def\texit@debug@node@color{1 1 0}

\protected\def\texit@debug@draw@page#1#2#3{%
  \pdfsave
  \pdfliteral{%
    .996264 0 0 .996264 0 0 cm
    q
    1 0 0 1 0 -\strip@pt\dimexpr\ht#1+\dp#1\relax\space cm
    .25 w
    #2
    Q
    .3 w 1 0 1 rg 1 0 1 RG
    0 0 \strip@pt\wd#1\space -\strip@pt\dimexpr\ht#1+\dp#1\relax\space re S
  }%
  \nointerlineskip \vbox to\z@{\vss\hbox{\texit@debugfont {\texit@debugfontbold#3}\enspace(\the\ht#1 + \the\dp#1) x \the\wd#1}\kern\p@}
  \pdfrestore
}

\protected\def\texit@debug@draw@box#1#2#3{%
  \global\texit@debug@boxes=\expanded{{%
      \the\texit@debug@boxes
      1 0 0 1 0 \strip@pt\dimen@i\space cm
      q
      #1 rg #1 RG #2
      0 0 \strip@pt\wd#3\space \strip@pt\dimexpr\ht#3+\dp#3\relax\space re S
      q
      [1.25] 0 d
      0 \strip@pt\dp#3\space m
      \strip@pt\wd#3\space \strip@pt\dp#3\space l S
      Q
      Q
      1 0 0 1 0 \strip@pt\dimexpr\ht#3+\dp#3\relax\space cm
  }}%
}

\protected\def\texit@debug@draw@end#1#2#3#4{%
  \pdfsave
  \pdfliteral{#1 rg #1 RG}%
  \nointerlineskip \moveright\dimexpr#3\relax \vbox to\z@{%
    \dimen@=\dimexpr#4\relax
    \kern-\dimen@
    \kern-4\p@
    \vbox to4\p@{\hbox to\z@{\hss \texit@debugfont #2\hss}\vss}
    \pdfliteral{%
      .996264 0 0 .996264 0 0 cm
      1.25 w 1 j
      0 0 m 0 0.01 l s
    }%
    \vss
  }%
  \pdfrestore
}

\protected\def\texit@debug@footline{%
  \pdfsave
  \pdfliteral{1 0 1 rg}%
  \nointerlineskip \moveright\p@ \vbox to\z@{%
    \vss
    \hbox{\texit@debugfont
      {\texit@debugfontbold
        \ifdefined\directlua
        LuaTeX v\number\luatexversion:\number\luatexrevision
        \else
        pdfTeX v\number\pdftexversion:\number\pdftexrevision
        \fi}\quad
      {\texit@debugfontbold texit.cls v\texit@version}\quad
      id: {\texit@debugfontbold \jobname}\quad
      \iftexit@textgradient
      textgradient: {\texit@debugfontbold
        \texit@textgradient\space (\texit@textgradientangle \iftexit@textgradientglow, glow\fi)}%
      \else
      textcolor: {\texit@debugfontbold \texit@textcolor}%
      \fi \quad
      \iftexit@pagegradient
      pagegradient: {\texit@debugfontbold
        \texit@pagegradient\space (\texit@pagegradientangle)}%
      \else
      pagecolor: {\texit@debugfontbold \texit@pagecolor}%
      \fi \quad
      {\texit@debugfontbold \iftexit@singlepage single\else multi\fi page}\quad
      minpagewidth: {\texit@debugfontbold \ifdim\texit@minpagewidth<1sp None\else \the\texit@minpagewidth \fi}\quad
      minhsize: {\texit@debugfontbold \ifdim\texit@minhsize<1sp None\else \the\texit@minhsize \fi}\quad
      maxhsize: {\texit@debugfontbold \ifdim\texit@maxhsize=\maxdimen None\else \the\texit@maxhsize \fi}\quad
      border: {\texit@debugfontbold \strip@pt\texit@border
        \if0%
        \ifdim\texit@extraborder@T=\z@ \else 1\fi
        \ifdim\texit@extraborder@L=\z@ \else 1\fi
        \ifdim\texit@extraborder@B=\z@ \else 1\fi
        \ifdim\texit@extraborder@R=\z@ \else 1\fi
        0\else
        +(\strip@pt\texit@extraborder@T T, \strip@pt\texit@extraborder@L L, \strip@pt\texit@extraborder@B B, \strip@pt\texit@extraborder@R R)%
        \fi
        pt%
      }%
    }%
    \kern\p@
  }%
  \pdfrestore
}



% Patches to LaTeX2e for better OTR compatibility

% Lists force full width
\edef\endtrivlist{\unexpanded\expandafter{\endtrivlist}\penalty\texit@realwd@forcehsize}

% Minimal float handling

\newinsert\texit@topins \count\texit@topins=1000 \skip\texit@topins=\textfloatsep \dimen\texit@topins=\maxdimen
\newinsert\texit@botins \count\texit@botins=1000 \skip\texit@botins=\textfloatsep \dimen\texit@botins=\maxdimen

\newif\iftexit@floating

\def\texit@parsefloat#1{%
  \let\next=\relax
  \ifx\@nnil#1 \else
  \if!#1
  \let\next=\texit@parsefloat
  \else
  \let\next=\remove@to@nnil
  \if t#1
  \texit@floatingtrue
  \let\texit@floatclass=\texit@topins
  \else \if b#1
  \texit@floatingtrue
  \let\texit@floatclass=\texit@botins
  \fi \fi
  \fi
  \fi
  \next
}

\def\@xfloat#1[#2]{%
  \@nodocument
  \def\@captype{#1}%
  \texit@parsefloat#2\@nnil
  \ifinner \@parmoderr \fi
  \setbox\z@=\color@vbox
  \normalcolor
  \vbox\bgroup
  \hsize\columnwidth
  \@parboxrestore
  \@floatboxreset
}%
\def\end@float{\@endfloatbox}
\def\@endfloatbox{%
  \par\vskip\z@skip
  \@minipagefalse
  \outer@nobreak
  \egroup
  \color@endbox
  \iftexit@floating
  \insert\texit@floatclass{%
    \splittopskip=0pt
    \splitmaxdepth=\maxdimen
    \box\z@
    \vskip\@fpsep
  }%
  \else
  \vskip\intextsep
  \box\z@
  \vskip\intextsep
  \fi
}

\def\fps@figure{h}
\def\fps@table{h}

% 5pt-high footnoterule
\protected\def\footnoterule{%
  \kern-\p@
  \hrule\@width\footnoterulewd \@height.4\p@
  \kern5.6\p@
}
\def\footnoteruleht{5\p@}
\def\footnoterulewd{.4\columnwidth}

\renewcommand\marginpar[2][]{\ClassError{texit}{Not supported}{}}

% COLOURSCHEME
% \newcommand*\Template{
% \definecolor{fg}      {RGB}{}
% \definecolor{bg}      {RGB}{}
% \definecolor{black}      {RGB}{}
% \definecolor{red}      {RGB}{}
% \definecolor{green}      {RGB}{}
% \definecolor{yellow}    {RGB}{}
% \definecolor{blue}      {RGB}{}
% \definecolor{magenta}    {RGB}{}
% \definecolor{cyan}      {RGB}{}
% \definecolor{white}      {RGB}{}
% \definecolor{brightblack}  {RGB}{}
% \definecolor{brightred}    {RGB}{}
% \definecolor{brightgreen}  {RGB}{}
% \definecolor{brightyellow}  {RGB}{}
% \definecolor{brightblue}  {RGB}{}
% \definecolor{brightmagenta}  {RGB}{}
% \definecolor{brightcyan}  {RGB}{}
% \definecolor{brightwhite}  {RGB}{}
% }
\definecolor{trueblack}{RGB}{0,0,0}
\definecolor{truewhite}{RGB}{255,255,255}


% Apple Colours
\newcommand*\Apple{%
  % REDS
  \newcommand*\applereds{lavender,bubblegum,carnation,salmon,
  maraschino,cayenne,mocha,tangerine,cantaloupe,plum,maroon}
  \definecolor{lavender}    {RGB}{215,131,255}
  \definecolor{bubblegum}    {RGB}{255,133,255}
  \definecolor{carnation}    {RGB}{255,138,216}
  \definecolor{salmon}    {RGB}{255,126,121}
  \definecolor{maraschino}  {RGB}{255, 38,  0}
  \definecolor{cayenne}    {RGB}{148, 17,  0}
  \definecolor{mocha}      {RGB}{148, 82,  0}
  \definecolor{tangerine}    {RGB}{255,147,  0}
  \definecolor{cantaloupe}  {RGB}{255,212,121}
  \definecolor{plum}      {RGB}{148, 33,147}
  \definecolor{maroon}    {RGB}{148, 23, 81}
  % GREENS
  \newcommand*\applegreens{asparagus,fern,teal,flora,honeydew,seafoam}
  \definecolor{asparagus}    {RGB}{146,144,  0}
  \definecolor{fern}      {RGB}{ 79,143,  0}
  \definecolor{teal}      {RGB}{  0,145,147}
  \definecolor{flora}      {RGB}{115,250,121}
  \definecolor{honeydew}    {RGB}{212,251,121}
  \definecolor{seafoam}    {RGB}{  0,250,146}
  % BLUES
  \newcommand*\appleblues{turquoise,ocean,midnight,eggplant,aqua,
  blueberry,spindrift,orchid}
  \definecolor{turquoise}    {RGB}{  0,253,255}
  \definecolor{ocean}      {RGB}{  0, 84,147}
  \definecolor{midnight}    {RGB}{  1, 25,147}
  \definecolor{eggplant}    {RGB}{ 83, 27,147}
  \definecolor{aqua}      {RGB}{  0,150,255}
  \definecolor{blueberry}    {RGB}{  4, 51,255}
  \definecolor{spindrift}    {RGB}{115,252,214}
  \definecolor{orchid}    {RGB}{122,129,255}
  % GREYSCALE
  \newcommand*\applegreys{liquorice,tungsten,iron,tin,silver,magnesium,mercury}
  \definecolor{liquorice}    {RGB}{  0,  0,  0}
  \definecolor{tungsten}    {RGB}{ 66, 66, 66}
  \definecolor{iron}      {RGB}{ 94, 94, 94}
  \definecolor{tin}      {RGB}{145,145,145}
  \definecolor{silver}    {RGB}{163,163,163}
  \definecolor{magnesium}    {RGB}{192,192,192}
  \definecolor{mercury}    {RGB}{235,235,235}
  \colorlet{bg}        {mercury}    
  \colorlet{fg}        {liquorice}  
  \colorlet{black}      {tungsten}    
  \colorlet{red}        {maraschino}  
  \colorlet{green}      {fern}      
  \colorlet{yellow}      {cantaloupe}  
  \colorlet{blue}        {blueberry}  
  \colorlet{magenta}      {plum}      
  \colorlet{cyan}        {seafoam}    
  \colorlet{white}      {magnesium}  
  \colorlet{brightblack}    {liquorice}  
  \colorlet{brightred}    {cayenne}    
  \colorlet{brightgreen}    {flora}    
  \colorlet{brightyellow}    {mocha}    
  \colorlet{brightblue}    {midnight}    
  \colorlet{brightmagenta}  {maroon}    
  \colorlet{brightcyan}    {teal}      
  \colorlet{brightwhite}    {mercury}
  \colorlet{texitpagecolor}{bg}
  \colorlet{texittextcolor}{fg}
  \texit@processall
}

\newcommand*\AyuMirage{
  \definecolor{fg}      {RGB}{217,215,206}
  \definecolor{bg}      {RGB}{ 33, 39, 51}
  \definecolor{black}      {RGB}{ 25, 30, 42}
  \definecolor{red}      {RGB}{237,130,116}
  \definecolor{green}      {RGB}{166,204,112}
  \definecolor{yellow}    {RGB}{250,208,123}
  \definecolor{blue}      {RGB}{109,203,250}
  \definecolor{magenta}    {RGB}{207,186,250}
  \definecolor{cyan}      {RGB}{144,225,198}
  \definecolor{white}      {RGB}{199,199,199}
  \definecolor{brightblack}  {RGB}{104,104,104}
  \definecolor{brightred}    {RGB}{242,135,121}
  \definecolor{brightgreen}  {RGB}{186,230,126}
  \definecolor{brightyellow}  {RGB}{255,213,128}
  \definecolor{brightblue}  {RGB}{115,208,255}
  \definecolor{brightmagenta}  {RGB}{212,191,255}
  \definecolor{brightcyan}  {RGB}{149,230,203}
  \definecolor{brightwhite}  {RGB}{255,255,255}
  \colorlet{texitpagecolor}{bg}
  \colorlet{texittextcolor}{fg}
  \texit@processall
}
\newcommand\beshirilari{%
  \definecolor{bg}      {RGB}{220,238,200} % rgb(220,238,200)
  \definecolor{fg}      {RGB}{ 23, 63, 79} % rgb( 23, 63, 79)
  \definecolor{black}      {RGB}{ 23, 63, 79} % rgb( 23, 63, 79)
  \definecolor{red}      {RGB}{220, 53, 69} % rgb(220, 53, 69)
  \definecolor{green}      {RGB}{115,186, 37} % rgb(115,186, 37)
  \definecolor{yellow}    {RGB}{255,193,  7} % rgb(255,193,  7)
  \definecolor{blue}      {RGB}{ 56,173,226} % rgb( 56,173,226)
  \definecolor{magenta}    {RGB}{ 26,173,149} % rgb( 26,173,149)
  \definecolor{cyan}      {RGB}{ 74,192,180} % rgb( 74,192,180)
  \definecolor{white}      {RGB}{238,238,238} % rgb(238,238,238)
  \definecolor{brightblack}  {RGB}{ 54, 95,111} % rgb( 54, 95,111)
  \definecolor{brightred}    {RGB}{200,102, 99} % rgb(200,102, 99)
  \definecolor{brightgreen}  {RGB}{109,167, 65} % rgb(109,167, 65)
  \definecolor{brightyellow}  {RGB}{255,193,  7} % rgb(255,193,  7)
  \definecolor{brightblue}  {RGB}{ 33,164,223} % rgb( 33,164,223)
  \definecolor{brightmagenta}  {RGB}{129,193, 59} % rgb(129,193, 59)
  \definecolor{brightcyan}  {RGB}{  0,164,137} % rgb(  0,164,137)
  \definecolor{brightwhite}  {RGB}{255,255,255} % rgb(255,255,255)
  \colorlet{texitpagecolor}{bg}
  \colorlet{texittextcolor}{fg}
  \texit@processall
}
\newcommand*\Breeze{
  \definecolor{bg}      {RGB}{ 30, 34, 41}
  \definecolor{fg}      {RGB}{187,187,187}
  \definecolor{black}      {RGB}{ 30, 34, 41}
  \definecolor{red}      {RGB}{237, 21, 21}
  \definecolor{green}      {RGB}{ 68,133, 58}
  \definecolor{yellow}    {RGB}{246,116,  0}
  \definecolor{blue}      {RGB}{ 29,153,243}
  \definecolor{magenta}    {RGB}{155, 89,182}
  \definecolor{cyan}      {RGB}{ 26,188,156}
  \definecolor{white}      {RGB}{252,252,252}
  \definecolor{brightblack}  {RGB}{127,140,141}
  \definecolor{brightred}    {RGB}{192, 57, 43}
  \definecolor{brightgreen}  {RGB}{ 85,166, 73}
  \definecolor{brightyellow}  {RGB}{253,188, 75}
  \definecolor{brightblue}  {RGB}{ 61,174,233}
  \definecolor{brightmagenta}  {RGB}{142, 67,173}
  \definecolor{brightcyan}  {RGB}{ 22,160,133}
  \definecolor{brightwhite}  {RGB}{255,255,255}
  \colorlet{texitpagecolor}{bg}
  \colorlet{texittextcolor}{fg}
  \texit@processall
}
\newcommand*\Chameleon{
  \definecolor{fg}      {RGB}{220,238,200}
  \definecolor{bg}      {RGB}{ 23, 63, 79}
  \definecolor{black}      {RGB}{ 23, 63, 79}
  \definecolor{red}      {RGB}{220, 53, 69}
  \definecolor{green}      {RGB}{115,186, 37}
  \definecolor{yellow}    {RGB}{255,193,  7}
  \definecolor{blue}      {RGB}{ 56,173,226}
  \definecolor{magenta}    {RGB}{ 26,173,149}
  \definecolor{cyan}      {RGB}{ 74,192,180}
  \definecolor{white}      {RGB}{238,238,238}
  \definecolor{brightblack}  {RGB}{ 54, 95,111}
  \definecolor{brightred}    {RGB}{200,102, 99}
  \definecolor{brightgreen}  {RGB}{109,167, 65}
  \definecolor{brightyellow}  {RGB}{255,193,  7}
  \definecolor{brightblue}  {RGB}{ 33,164,223}
  \definecolor{brightmagenta}  {RGB}{129,193, 59}
  \definecolor{brightcyan}  {RGB}{  0,164,137}
  \definecolor{brightwhite}  {RGB}{255,255,255}
  \colorlet{texitpagecolor}{bg}
  \colorlet{texittextcolor}{fg}
  \texit@processall
}
\newcommand*\Dracula{
  \definecolor{bg}      {RGB}{ 40, 42, 54}
  \definecolor{fg}      {RGB}{248,248,242}
  \definecolor{black}      {RGB}{ 33, 34, 44}
  \definecolor{red}      {RGB}{255, 85, 85}
  \definecolor{green}      {RGB}{ 80,250,123}
  \definecolor{yellow}    {RGB}{241,250,140}
  \definecolor{blue}      {RGB}{189,147,249}
  \definecolor{magenta}    {RGB}{255,121,198}
  \definecolor{cyan}      {RGB}{139,233,253}
  \definecolor{white}      {RGB}{248,248,242}
  \definecolor{brightblack}  {RGB}{ 98,114,164}
  \definecolor{brightred}    {RGB}{255,110,110}
  \definecolor{brightgreen}  {RGB}{105,255,148}
  \definecolor{brightyellow}  {RGB}{255,255,165}
  \definecolor{brightblue}  {RGB}{214,172,255}
  \definecolor{brightmagenta}  {RGB}{255,146,223}
  \definecolor{brightcyan}  {RGB}{164,255,255}
  \definecolor{brightwhite}  {RGB}{255,255,255}
  \colorlet{texitpagecolor}{bg}
  \colorlet{texittextcolor}{fg}
  \texit@processall
}

\newcommand*\Google{
  \definecolor{fg}      {RGB}{ 55, 59, 65}
  \definecolor{bg}      {RGB}{255,255,255}
  \definecolor{black}      {RGB}{ 29, 31, 33}
  \definecolor{red}      {RGB}{204, 52, 43}
  \definecolor{green}      {RGB}{ 25,136, 68}
  \definecolor{yellow}    {RGB}{251,169, 34}
  \definecolor{blue}      {RGB}{ 57,113,237}
  \definecolor{magenta}    {RGB}{163,106,199}
  \definecolor{cyan}      {RGB}{ 57,113,237}
  \definecolor{white}      {RGB}{197,200,198}
  \definecolor{brightblack}  {RGB}{150,152,150}
  \definecolor{brightred}    {RGB}{204, 52, 43}
  \definecolor{brightgreen}  {RGB}{ 25,136, 68}
  \definecolor{brightyellow}  {RGB}{251,169, 34}
  \definecolor{brightblue}  {RGB}{ 57,113,237}
  \definecolor{brightmagenta}  {RGB}{163,106,199}
  \definecolor{brightcyan}  {RGB}{ 57,113,237}
  \definecolor{brightwhite}  {RGB}{255,255,255}
  \colorlet{texitpagecolor}{bg}
  \colorlet{texittextcolor}{fg}
  \texit@processall
}
\newcommand*\Gruvbox{
  \definecolor{bg}      {RGB}{ 40, 40, 40}
  \definecolor{fg}      {RGB}{235,219,178}
  \definecolor{black}      {RGB}{ 40, 40, 40}
  \definecolor{red}      {RGB}{204, 36, 29}
  \definecolor{green}      {RGB}{152,151, 26}
  \definecolor{yellow}    {RGB}{215,153, 33}
  \definecolor{blue}      {RGB}{ 69,133,136}
  \definecolor{magenta}    {RGB}{177, 98,134}
  \definecolor{cyan}      {RGB}{104,157,106}
  \definecolor{white}      {RGB}{168,153,132}
  \definecolor{brightblack}  {RGB}{146,131,116}
  \definecolor{brightred}    {RGB}{251, 73, 52}
  \definecolor{brightgreen}  {RGB}{184,187, 38}
  \definecolor{brightyellow}  {RGB}{250,189, 47}
  \definecolor{brightblue}  {RGB}{131,165,152}
  \definecolor{brightmagenta}  {RGB}{211,134,155}
  \definecolor{brightcyan}  {RGB}{142,192,124}
  \definecolor{brightwhite}  {RGB}{235,219,178}
  \colorlet{texitpagecolor}{bg}
  \colorlet{texittextcolor}{fg}
  \texit@processall
}
\newcommand*\Moe{
  \definecolor{bg}      {RGB}{ 19, 19, 19}
  \definecolor{fg}      {RGB}{120,140,161}
  \definecolor{black}      {RGB}{ 43, 43, 43}
  \definecolor{red}      {RGB}{255, 88, 88}
  \definecolor{green}      {RGB}{ 47,190,188}
  \definecolor{yellow}    {RGB}{ 59,175,228}
  \definecolor{blue}      {RGB}{255, 88, 88}
  \definecolor{magenta}    {RGB}{198,121,221}
  \definecolor{cyan}      {RGB}{255, 99,118}
  \definecolor{white}      {RGB}{ 62, 79, 88}
  \definecolor{brightblack}  {RGB}{ 38, 38, 38}
  \definecolor{brightred}    {RGB}{255, 89,125}
  \definecolor{brightgreen}  {RGB}{ 47,190,188}
  \definecolor{brightyellow}  {RGB}{ 54,158,207}
  \definecolor{brightblue}  {RGB}{255, 99,118}
  \definecolor{brightmagenta}  {RGB}{198,120,221}
  \definecolor{brightcyan}  {RGB}{255, 99,118}
  \definecolor{brightwhite}  {RGB}{ 50, 64, 71}
  \colorlet{texitpagecolor}{bg}
  \colorlet{texittextcolor}{fg}
  \texit@processall
}
\newcommand*\Nord{
  \definecolor{bg}      {RGB}{ 46, 52, 64}
  \definecolor{fg}      {RGB}{216,222,233}
  \definecolor{black}      {RGB}{ 59, 66, 82}
  \definecolor{red}      {RGB}{191, 97,106}
  \definecolor{green}      {RGB}{163,190,140}
  \definecolor{yellow}    {RGB}{235,203,139}
  \definecolor{blue}      {RGB}{129,161,193}
  \definecolor{magenta}    {RGB}{180,142,173}
  \definecolor{cyan}      {RGB}{136,192,208}
  \definecolor{white}      {RGB}{229,233,240}
  \definecolor{brightblack}  {RGB}{ 76, 86,106}
  \definecolor{brightred}    {RGB}{191, 97,106}
  \definecolor{brightgreen}  {RGB}{163,190,140}
  \definecolor{brightyellow}  {RGB}{235,203,139}
  \definecolor{brightblue}  {RGB}{129,161,193}
  \definecolor{brightmagenta}  {RGB}{180,142,173}
  \definecolor{brightcyan}  {RGB}{143,188,187}
  \definecolor{brightwhite}  {RGB}{236,239,244}
  \colorlet{texitpagecolor}{bg}
  \colorlet{texittextcolor}{fg}
  \texit@processall

}
\newcommand*\Radical{% Original Theme
  \definecolor{bg}      {RGB}{ 20, 19, 34}
  \definecolor{fg}      {RGB}{160,205,207}
  \definecolor{black}      {RGB}{ 31, 33,111}
  \definecolor{red}      {RGB}{255, 68,138}
  \definecolor{green}      {RGB}{211,255, 63}
  \definecolor{yellow}    {RGB}{255,252,114}
  \definecolor{blue}      {RGB}{ 76, 77,228}
  \definecolor{magenta}    {RGB}{248, 83,177}
  \definecolor{cyan}      {RGB}{178,246,252}
  \definecolor{white}      {RGB}{201,238,230}
  \definecolor{brightblack}  {RGB}{ 41, 12,173}
  \definecolor{brightred}    {RGB}{255, 50,111}
  \definecolor{brightgreen}  {RGB}{194,255,  0}
  \definecolor{brightyellow}  {RGB}{247,211, 56}
  \definecolor{brightblue}  {RGB}{ 12, 25,174}
  \definecolor{brightmagenta}  {RGB}{208, 37,131}
  \definecolor{brightcyan}  {RGB}{160,254,247}
  \definecolor{brightwhite}  {RGB}{197,255,240}
  \colorlet{texitpagecolor}{bg}
  \colorlet{texittextcolor}{fg}
  \texit@processall
}
\newcommand*\Shizuka{% Dress up Darling
  \definecolor{bg}      {RGB}{ 21, 18, 25}
  \definecolor{fg}      {RGB}{234,220,227}
  \definecolor{black}      {RGB}{ 21, 18, 25}
  \definecolor{red}      {RGB}{161,142,149}
  \definecolor{green}      {RGB}{209,149,164}
  \definecolor{yellow}    {RGB}{201,159,156}
  \definecolor{blue}      {RGB}{224,195,170}
  \definecolor{magenta}    {RGB}{206,179,215}
  \definecolor{cyan}      {RGB}{225,192,215}
  \definecolor{white}      {RGB}{234,220,227}
  \definecolor{brightblack}  {RGB}{156,144,151}
  \definecolor{brightred}    {RGB}{161,142,149}
  \definecolor{brightgreen}  {RGB}{209,149,164}
  \definecolor{brightyellow}  {RGB}{201,159,156}
  \definecolor{brightblue}  {RGB}{224,195,170}
  \definecolor{brightmagenta}  {RGB}{206,179,215}
  \definecolor{brightcyan}  {RGB}{225,192,215}
  \definecolor{brightwhite}  {RGB}{234,220,227}
  \colorlet{texitpagecolor}{bg}
  \colorlet{texittextcolor}{fg}
  \texit@processall
}
\newcommand*\SolarisedDark{
  %|SOLARIZED|HEX    |16/8|TERMCOL  |XTERM/HEX  |sRGB       |
  %|---------|-------|----|-------  |-----------|-----------|
  %|base03   |#002b36| 8/4|brblack  |234 #1c1c1c|  0  43  54|
  %|base02   |#073642| 0/4|black    |235 #262626|  7  54  66|
  %|base01   |#586e75|10/7|brgreen  |240 #4e4e4e| 88 110 117|
  %|base00   |#657b83|11/7|bryellow |241 #585858|101 123 131|
  %|base0    |#839496|12/6|brblue   |244 #808080|131 148 150|
  %|base1    |#93a1a1|14/4|brcyan   |245 #8a8a8a|147 161 161|
  %|base2    |#eee8d5| 7/7|white    |254 #d7d7af|238 232 213|
  %|base3    |#fdf6e3|15/7|brwhite  |230 #ffffd7|253 246 227|
  %|yellow   |#b58900| 3/3|yellow   |136 #af8700|181 137   0|
  %|orange   |#cb4b16| 9/3|brred    |166 #d75f00|203  75  22|
  %|red      |#dc322f| 1/1|red      |160 #d70000|220  50  47|
  %|magenta  |#d33682| 5/5|magenta  |125 #af005f|211  54 130|
  %|violet   |#6c71c4|13/5|brmagenta| 61 #5f5faf|108 113 196|
  %|blue     |#268bd2| 4/4|blue     | 33 #0087ff| 38 139 210|
  %|cyan     |#2aa198| 6/6|cyan     | 37 #00afaf| 42 161 152|
  %|green    |#859900| 2/2|green    | 64 #5f8700|133 153   0|
  \definecolor{bg}      {RGB}{  0, 43, 54}
  \definecolor{fg}      {RGB}{147,161,161}
  \definecolor{black}      {RGB}{  7, 54, 66}
  \definecolor{red}      {RGB}{220, 50, 45}
  \definecolor{green}      {RGB}{133,153,  0}
  \definecolor{yellow}    {RGB}{181,137,  0}
  \definecolor{blue}      {RGB}{ 38,139,210}
  \definecolor{magenta}    {RGB}{211, 54,130}
  \definecolor{cyan}      {RGB}{ 42,161,152}
  \definecolor{white}      {RGB}{238,232,213}
  \definecolor{brightblack}  {RGB}{  0, 43, 54}
  \definecolor{brightred}    {RGB}{  7, 54, 66}
  \definecolor{brightgreen}  {RGB}{ 88,110,117}
  \definecolor{brightyellow}  {RGB}{101,123,131}
  \definecolor{brightblue}  {RGB}{131,148,150}
  \definecolor{brightmagenta}  {RGB}{147,161,161}
  \definecolor{brightcyan}  {RGB}{238,232,213}
  \definecolor{brightwhite}  {RGB}{253,246,227}
  \colorlet{texitpagecolor}{bg}
  \colorlet{texittextcolor}{fg}
  \texit@processall
}
\newcommand*\SolarisedLight{
  \SolarisedDark
  \definecolor{bg}      {RGB}{253,246,227}
  \definecolor{fg}      {RGB}{  7, 54, 66}
  \colorlet{texitpagecolor}{bg}
  \colorlet{texittextcolor}{fg}
  \texit@processall
}
\newcommand*\SUSEDark{
  % https://brand.suse.com/brand-system/color-palette
  \definecolor{bg}      {RGB}{ 12, 50, 44}
  \definecolor{fg}      {RGB}{255,255,255}
  \definecolor{black}      {RGB}{ 12, 50, 44}
  \definecolor{red}      {RGB}{254,124, 63}
  \definecolor{green}      {RGB}{ 48,186,120}
  \definecolor{yellow}    {RGB}{255,193,  7}
  \definecolor{blue}      {RGB}{ 36, 83,255}
  \definecolor{magenta}    {RGB}{ 25, 32,114}
  \definecolor{cyan}      {RGB}{144,235,205}
  \definecolor{white}      {RGB}{247,247,247}
  \definecolor{brightblack}  {RGB}{ 12, 50, 44}
  \definecolor{brightred}    {RGB}{254,124, 63}
  \definecolor{brightgreen}  {RGB}{ 48,186,120}
  \definecolor{brightyellow}  {RGB}{255,193,  7}
  \definecolor{brightblue}  {RGB}{ 36, 83,255}
  \definecolor{brightmagenta}  {RGB}{ 25, 32,114}
  \definecolor{brightcyan}  {RGB}{144,235,205}
  \definecolor{brightwhite}  {RGB}{255,255,255}
  \colorlet{PineGreen}{bg}
  \colorlet{JungleGreen}{green}
  \colorlet{MidnightBlue}{magenta}
  \colorlet{WaterholeBlue}{blue}
  \colorlet{Mint}{cyan}
  \colorlet{Persimmon}{red}
  \colorlet{Fog}{white}
  \colorlet{texitpagecolor}{bg}
  \colorlet{texittextcolor}{fg}
  \texit@processall
}
\newcommand*\SUSELight{
  \SUSEDark
  \definecolor{bg}      {RGB}{247,247,247}
  \definecolor{fg}      {RGB}{  0,  0,  0}
  \colorlet{texitpagecolor}{bg}
  \colorlet{texittextcolor}{fg}
  \texit@processall
}
\newcommand*\Transrights{% Original Theme
  \definecolor{bg}      {RGB}{ 29, 31, 33}
  \definecolor{fg}      {RGB}{149,152,150}
  \definecolor{black}      {RGB}{ 29, 31, 33}
  \definecolor{red}      {RGB}{196,121,162}
  \definecolor{green}      {RGB}{237,165,205}
  \definecolor{yellow}    {RGB}{214,199,232}
  \definecolor{blue}      {RGB}{214,199,232}
  \definecolor{magenta}    {RGB}{109,130,209}
  \definecolor{cyan}      {RGB}{154,199,248}
  \definecolor{white}      {RGB}{255,255,255}
  \definecolor{brightwhite}  {RGB}{255,255,255}
  \definecolor{brightblack}  {RGB}{102,102,102}
  \definecolor{brightred}    {RGB}{196,121,162}
  \definecolor{brightgreen}  {RGB}{237,165,205}
  \definecolor{brightyellow}  {RGB}{214,199,232}
  \definecolor{brightblue}  {RGB}{214,199,232}
  \definecolor{brightmagenta}  {RGB}{109,130,209}
  \definecolor{brightcyan}  {RGB}{154,199,248}
  \colorlet{texitpagecolor}{bg}
  \colorlet{texittextcolor}{fg}
  \texit@processall
}

% TeXit gradient switches
%% Rainbows
\NewDocumentCommand{\rainbow}{ O{0} }{%
  \texitset{textgradient={FF3333,FF5500,909000,667B3F,009000,60D0A0,009B9B,6060F0,9060D0,D060D0,ff6db3}, textgradientangle=#1, pagecolor=truewhite}
}
\NewDocumentCommand{\rainbowtwo}{ O{0} }{%
  \texitset{textgradient={6060F0,6090D0,009B9B,60D0A0,009000,667B3F,909000,FF5500,FF3333,ff6db3,9060D0}, textgradientangle=#1, pagecolor=trueblack}
}
  
\NewDocumentCommand{\rainbowbg}{ O{0} }{%
  \texitset{pagegradient={FF3333,FF5500,909000,667B3F,009000,60D0A0,009B9B,6060F0,9060D0,D060D0,ff6db3}, pagegradientangle=#1, textcolor=trueblack}
}
\NewDocumentCommand{\rainbowbgtwo}{ O{0} }{%
  \texitset{pagegradient={6060F0,6090D0,009B9B,60D0A0,009000,667B3F,909000,FF3333,ff6db3,9060D0}, pagegradientangle=#1, textcolor=truewhite}
}
\let\rb=\rainbow
\let\wrb=\rainbowtwo
\let\bgrb=\rainbowbg
\let\bgwrb=\rainbowbgtwo

\NewDocumentCommand{\ShowThemes}{}{%
  \par\noindent{\bfseries Colorschemes}\par\smallskip
  \ttfamily Apple, AyuMirage, Breeze, Chameleon, Dracula, Google, Gruvbox,
  Moe, Nord, Radical, Shizuka, SolarisedDark, SolarisedLight,
  SUSEDark, SUSELight, Transrights, beshirilari\par\medskip
  \noindent{\bfseries Gradients}\par\smallskip
  rainbow, rainbowtwo, rainbowbg, rainbowbgtwo\par
}

% COLORPROFILE
% Colour Profiling
\newcommand\colorprofile{\@ifstar{\CP@vals}{\CP@table}}
\long\protected\def\CP@vals#1{% Tabulated Conversion Profile
	\extractcolorspecs{#1}{\CP@mdl}{\CP@clr}
	\def\c@html{\convertcolorspec{\CP@mdl}{\CP@clr}{HTML}\tmp\tmp}
	\def\c@rgb{\convertcolorspec{\CP@mdl}{\CP@clr}{RGB}\tmp\tmp}
	\def\c@cmyk{\convertcolorspec{\CP@mdl}{\CP@clr}{cmyk}\tmp\tmp}
	\vbox{\offinterlineskip\footnotesize\tt
		\halign{\color{fg}
			\hfil##:~&\vtop{\parindent0em\hangindent0em \strut##}\cr
			HTML&\#\c@html\cr RGB &[\c@rgb]\cr CMYK&[\c@cmyk]\cr
		}
	}
}
\long\protected\def\CP@table#1{% Visualisation Table in 000/FFF
	\extractcolorspecs{#1}{\CP@mdl}{\CP@clr}
	\definecolor{w}{HTML}{FFFFFF} \definecolor{b}{HTML}{000000}
	\def\c@html{\convertcolorspec{\CP@mdl}{\CP@clr}{HTML}\tmp\tmp}
	\def\c@rgb{\convertcolorspec{\CP@mdl}{\CP@clr}{RGB}\tmp\tmp}
	\def\c@cmyk{\convertcolorspec{\CP@mdl}{\CP@clr}{cmyk}\tmp\tmp}
	\def\c@html@fmt{\Large\ttfamily\c@html}
	\vbox{\offinterlineskip\footnotesize\tt\hsize=18em
		\halign{\hss\strut##\strut&\strut##\strut\hss\cr
			\fcolorbox{w}{w}{\color{#1}\c@html@fmt}&
			\fcolorbox{w}{b}{\color{#1}\c@html@fmt}\cr
			\fcolorbox{w}{#1}{\color{w}\c@html@fmt}&
			\fcolorbox{w}{#1}{\color{b}\c@html@fmt}\cr
		}
	}
}
% Visualise a colour for a given HTML HEX.
\newcommand\htmlcolor{\@ifstar{\HTML@vals}{\HTML@tab}}
\long\protected\def\HTML@tab#1{\definecolor{#1}{HTML}{#1}\colorprofile{#1}}
\long\protected\def\HTML@vals#1{\definecolor{#1}{HTML}{#1}\colorprofile*{#1}}

% Visualise the combination of colours for the current theme.
\newcommand{\themetable}[1]{%
	\begingroup\centering
	\def\out@fmt{{\footnotesize\ttfamily #1}}
	\def\normal{black,red,green,yellow,blue,magenta,cyan,white}
	\definecolor{b}{HTML}{000000}
	\foreach \x in \normal{
		\foreach \y in \normal{
			\fcolorbox{bg}{\x}{%
				\textcolor{\y}{\out@fmt}
				\textcolor{bright\y}{\out@fmt}}\hspace{-0.5em}
		}\vspace{-0.167em}

	}
	\foreach \x in \normal{
		\foreach \y in \normal{
			\fcolorbox{fg}{bright\x}{%
				\textcolor{\y}{\out@fmt}
				\textcolor{bright\y}{\out@fmt}}\hspace{-0.5em}
		}\vspace{-0.167em}

	}\endgroup
}
\let\ctable\themetable

% FONTTABLE
\long\def\fonttable#1{%
    \def\h@x{0, 1, 2, 3, 4, 5, 6, 7, 8, 9, A, B, C, D, E, F}
    \def\FT@query{#1}
    \font\FT@default=qx-lmtt8%
    \font\FT@selected=\FT@query%
    \def\FT@width{\the\dimexpr\hsize/17}%
    \vbox{\offinterlineskip
        \halign to \hsize{\tabskip=0em plus0em
            \hfil##&\hfil##\hfil\cr
            \multispan{2} \hfil{\color{magenta}\FT@selected\fontname\the\font}\hfil\cr\noalign{\bigskip}
            \makebox[\the\dimexpr\FT@width/2]{}&
            \foreach \j in \h@x{\makebox[\FT@width][c]{\FT@default\color{magenta}\char"22 \textcolor{magenta!25!bg}{x}\j}}\cr
        }\medskip{\color{bg!80!fg}\hrule}\smallskip
        \foreach \h in \h@x{%
            \mbox{\FT@default\color{magenta}\char"22 \h\textcolor{magenta!25!bg}{x}}\foreach\g in \h@x{%
                \makebox[\FT@width][c]{\FT@selected\color{fg}\char"\h\g}} \vskip.25em {\color{bg!80!fg}\hrule}\smallskip
        }
    }
}

% DRAW
% Tikz and Pgfplots
\RequirePackage{tikz}
\RequirePackage{pgfplots}
\pgfplotsset{%
	compat=1.18,
	tick label style = {font=\scriptsize},
	every axis label = {font=\footnotesize},
	legend style = {font=\footnotesize},
	label style = {font=\footnotesize}
}%
\pgfkeys{/pgf/number format/.cd,1000 sep={}}

\usepackage{tcolorbox}
\tcbuselibrary{most, breakable, listings}
\usepackage{multirow}
\usepackage{multicol}
\usepackage{colortbl} % add colour to tables

% ZZZ
% Must be loaded last.
\NewDocumentCommand{\ShowModules}{}{%
    \begingroup\par\noindent\ttfamily
    % Colourscheme
    \ifdefined\ShowThemes ShowThemes\par\medskip\else\fi
    % Typ
    \ifdefined\ShowTypDefs ShowTypDefs\par\medskip\else\fi
    \endgroup\par
}

\Radical
\leosays{pagecolor=bg, textcolor=fg}


\makeatother
